\section{Implementierung}

Die Implementierung dieser Applikation erfolgte wie gefordert in \textsl{C++}. Es handelt sich hierbei nicht nur um die Implementierung des bereits beschriebenen
Synchronisationsalgorithmus. Es musste auch die Netzwerkkommunikation zwischen den Programmen und die interne Kommunikation zwischen den einzelnen Komponenten
der Applikation realisiert werden.

\subsection{Algorithmus}

Der Algorithmus wurde implementiert, so dass er wie bereits beschrieben funktioniert. Der Großteil der Implementierung ist in der Datei 
\textit{src\slash file\_operator\slash sync\_system.cpp} zu finden, wobei Teile des Algorithmus, wie etwa die Signaturfunktionen, sich in anderen Modulen befinden.

Die \textbf{schwache Signatur} wurde einfach aus ihrer bereits beschriebenen mathematische Darstellung in \textsl{C++}-Code umgeschrieben. 

Für die \textbf{starke Signatur}, also \textit{MD5}, wurde \textit{OpenSSL} verwendet, welches diese Funktion leicht zu verwenden macht\cite{openssl}.

Auf die restliche Implementierung des Algorithmus hier nun konkret einzugehen würde aufgrund der Komplexität und Durchwachsenheit dieser Teile des Programms den Rahmen sprengen.

\subsection{Netzwerkkommunikation und Synchronisationsverlauf}

Die Kommunikation zwischen einem Synchronisations-Client und einem\\ Synchronisations-Server inklusive Verlauf des Synchronisationsprozesses sieht wie folgt aus:
\begin{enumerate}
\item Der Client initialisiert die Kommunikation, indem er eine Anfrage an den Server schickt, ihm eine Liste sämtlicher seiner synchronisierbaren Dateien zu schicken.
\item Der Server geht dieser Anfrage nach und schickt dem Client ein Liste sämtlicher Dateien.
\item Der Client vergleicht nun diese Liste mit seinen lokalen Dateien. Dateien, die er nicht kennt, werden vom Server angefordert, Dateien, von denen er weiß, dass sie bei ihm
      bereits gelöscht sind, werden dem Server entsprechen als zu löschen vermittelt. Bei Lokale Dateien, die nicht in der List des Servers aufscheinen, schickt der Client
      eine Anfrage, ob diese gelöscht wurden oder der Server sie vom Client braucht.  All restlich Dateien, die, die mit ihrem Namen bzw. Pfad sowohl beim Client als auch Server 
      vorhanden sind, werden auf Übereinstimmung bezüglich Zeitstempel und Gesamtsignatur überprüft. Für die Dateipaare, bei denen diese Wert unterschiedlich sind, wird der 
      Synchronisationsalgorithmus initialisiert und der Client schickt eine entsprechende Synchronisationsanfrage mit den schwachen Signaturen an den Server.
\item Der Server folgt den Nachrichten des Clients und sendet entsprechende Antworten zurück (z.B. eine angeforderte Datei). Bekommt der Server eine Synchronisationsanfrage
      so überprüft er die schwachen Signaturen mit der seiner lokalen Datei. Gibt es Übereinstimmungen, so schickt der Server ein Anfrage für die starken Signaturen der
      übereinstimmenden Blöcke an den Client. Abhängig von den Zeitstempeln der Dateien schickt der Server für die nicht übereinstimmenden Blöcke entweder seine Inhalte
      an den Client oder fragt die entsprechenden Inhalte beim Client an. (Der mit dem kleineren Zeitstempel, sprich der älteren Datei, muss sich die Daten holen, um
      den neueren Stand der Datei rekonstruieren zu können.)
\item Der Client schickt die geforderten starken Signaturen an den Server und, wenn gefordert, auch den Inhalt der nicht übereinstimmenden Blöcke bzw. speichert der Client
      die erhaltenen Korrekturdaten zur späteren Dateikonstruktion.
\item Der Server überprüft die starken Signaturen den Clients mit seinen eigenen. Wie bei den schwachen auch, schickt er für die nicht übereinstimmenden Blöcke die Korrekturdaten
      an den Client bzw. fordert sie an. Mit den übereinstimmenden Blöcken wird nichts mehr gemacht.
\item Der jenige, der die Korrekturdaten erhalten hat, (Client oder Server) muss nun die neue Datei rekonstruieren. Dafür wird aus Sicherheitsgründen eine temporäre Datei
      im Unterverzeichnis \textit{.sync} erstellt. Diese wird nun in der entsprechenden Reihenfolge nacheinander mit den Korrekturdaten befüllt und für die Stellen, für die
      es keine Korrekturdaten gibt, wird der Inhalt aus der alten, lokalen Datei verwendet, da diese an dieser Stelle gleich sein muss, zumindest laut den Signaturen, 
      sonst gebe es ja Korrekturdaten dafür. Am Ende wird die zusammengebaute Datei an ihren richtigen Ort verschoben und die alte somit verschoben.
\end{enumerate}

Die Netzwerkkommunikation ist mithilfe von \textit{Protocol Buffers}\cite{proto} für die Serialisierung der Nachrichten und \textit{Asio}\cite{asio} zur Übertragung 
der serialisierten Nachrichten realisiert.

Da \textit{Asio} eine Bibliothek zur textbasierten Netzwerkübertragung ist und \textit{Protocol Buffers} die Nachrichten in ein binäres Format serialisiert, kann es dazu kommen,
dass in den Nachrichten das Zeichen zur Trennung von Nachrichten und dem beenden des Leseflusses für \textit{Asio} vorkommt. In diesem Fall ist es das \textit{Newline}-Zeichen,
\verb|\n|. Da \textit{Protocol Buffers} dieses Zeichen im Serialisierungsformat durchaus benutzt, werden die serialisierten Nachrichten noch zusätzlich in Base 64 kodiert und
müssen dementsprechend am anderen Ende wieder dekodiert werden, noch bevor sie deserialisiert werden können. Dadurch werden Problem zwischen der textbasierten Netzwerkkommunikation
und der Serialisierung vermieden.

\subsection{Interne Kommunikation}

Die Applikation besteht aus mehreren Prozessen, die voneinander halbwegs unabhängig agieren sollten. Es sind drei Haupt-Prozesse, die für die Funktionstüchtigkeit der Applikation
verantwortlich sind:
\begin{itemize}
\item Der \textbf{Server} ist für die Kommunikation mit anderen Clients zuständig und startet für jede Verbindung einen eigenen Thread.

\item Der \textbf{Client} ist für die Kommunikation mit einem Server zuständig.

\item Der \textbf{File Operator} ist für den gesamten Synchronisationsablauf und der Kommunikation mit dem Dateisystem zuständig, sowohl Client- als auch Server-seitig. 
\end{itemize}

Neben diesen drei Hauptprozessen gibt es noch einen Prozess, welcher sich um \textbf{Kommandozeile} kümmert. \\

Die Kommunikation zwischen diesen Akteuren erfolgt über \textbf{Pipes}. Der File Operator bekommt das empfangende Ende einer Pipe und der Server und der Client (auch 
die Kommandozeile) bekommen das sendende Ende dieser Pipe, über welches sie Anfragen an den File Operator schicken können, welche dieser bearbeiten soll. Damit der File
Operator ihnen Antworten auf ihre Anfragen zurückschicken kann, schicken sie das sendende Ende einer Pipe mit, welche sich in ihrem Besitz befindet. Dieses kann der File Operator
benutzen, um ihnen Rückmeldungen zu schicken.

Um die Unterscheidung von sendenden und empfangenden Enden von Pipes zu ermöglichen, wurden Interfaces (rein abstrakte Klassen) und Klassenvererbung herangezogen.
Bei den sendenden und empfangenden Enden einer Pipe handelt sich um zwei Interfaces, welche von der Pipe implementiert werden.

\begin{figure}
    \centering
    \includegraphics[width=\textwidth]{images/pipe_uml.pdf}
    \caption{UML-Klassendiagramm von Pipe}
    \label{fig:uml}
\end{figure}

Wie aus dem Klassendiagramm in Abbildung \ref{fig:uml} ersichtlich ist, haben das sendende Ende, \verb|SendingPipe|, und das empfangende Ende, \verb|ReceivingPipe|, eine Menge aus 
gemeinsamen Operationen, welche im Interface \verb|Closable| definiert sind, ein Pipe ist nämlich schließbar. Was aus dem Diagramm auch noch ersichtlich ist, ist dass die beiden
Enden der Pipe und so auch die Pipe generell generisch für einen beliebigen Typ sind. Soll ein Prozess mit einer Pipe jetzt nur senden bzw. nur empfangen können, so bekommt
dieser nur die Schnittstelle \verb|SendingPipe| bzw. \verb|ReceivingPipe| zur Verfügung gestellt.

Schauen wir uns nun die Implementierung von \verb|Pipe| etwas genauer an, diese muss schließlich auch threadsafe sein.

\noindent\hrulefill\par
\begin{minipage}{\linewidth}
\begin{lstlisting}[language=C++, caption=Senden einer Nachricht ... bool send(T msg)]
std::lock_guard pipe_lck{pipe_mtx};
if (is_open()) {
    msgs.push(std::move(msg));
    receiving_finishable.notify_one();

    return true;
}
else {
    return false;
}  
\end{lstlisting}
\end{minipage}

Hie sehen wir zuerst die Funktion zum senden, \verb|bool send(T msg)|. Da es sich um eine schließbare Pipe handelt muss überprüft werden, ob die Pipe überhaupt noch offen ist.
Von Interesse ist, dass das Mutex für die Pipe bereits vor der Überprüfung, ob die Pipe offen ist, also noch bevor überhaupt entschieden wurde, ob gesendet werden kann, gesperrt 
wird. Dies hat damit zu tun, dass die Funktion \verb|close()|, zum Schließen der Pipe, auch dieses Mutex sperren muss. Durch das frühzeitige Sperren des Mutex beim Senden
wird vermieden, dass die Pipe nach der Abfrage, ob sie offen ist, aber noch vor dem eigentlichen Senden, geschlossen wird. Das gleiche sieht man auch beim Empfangen.
Ansonsten passiert nicht viel mehr außer, dass die Nachricht eben in die Queue, welche der interne Speicher der Pipe ist, gelegt wird und anschließend wird einer der aufs Empfangen
Wartenden benachrichtigt. Wenn das Senden erfolgreich war, sprich die Pipe offen war gibt es \verb|true| zurück, ansonsten \verb|false|.
Neben dieser Methode gibt es noch \verb|bool send(std::vector<T>& msgs)| zum Senden von Nachrichten. Diese funktioniert sehr ähnlich, kann jedoch mehrere Nachrichten nehmen
und legt sie alle gleichzeitig in der erhaltenen Reihenfolge in die Queue, sobald dies möglich ist.\\

\noindent\hrulefill\par
\begin{minipage}{\linewidth}
\begin{lstlisting}[language=C++, caption=Empfangen einer Nachricht ... std::optional<T> receive()]
std::unique_lock pipe_lck{pipe_mtx};
if (is_open()) {
    // warten, dass es etwas zum Empfangen gibt
    // oder die Pipe geschlossen wurde
    receiving_finishable.wait(
        pipe_lck, 
        [this](){ 
            return is_not_empty() || is_closed(); 
        }
    );

    if (is_open()) {
        T msg{std::move(msgs.front())};
        msgs.pop();

        return msg;
    }
    else {
        return std::nullopt;
    }
}
else {
    return std::nullopt;
}
\end{lstlisting}
\end{minipage}

Beim Empfangen mit \verb|std::optional<T> receive()| haben wir, wie bereits erwähnt, auch den Fall das als aller erstes die Mutex gesperrt wird und dann wird wieder überprüft, 
ob die Pipe offen ist. Ist die Pipe offen, wird mit der Condition Variable \verb|receiving_finishable| überprüft, ob es etwas zum Empfangen gibt und, falls nötig, darauf gewartet.
Ist das Warten vollendet wird nochmals überprüft, ob die Pipe offen ist, da das Warten auch durch das Schließen der Pipe beendet wird. Wenn ja, holt sich die Methode
die vorderste Nachricht aus der Queue, löscht sie aus der Queue raus und gibt sie an den Aufrufer der Methode zurück. Konnte keine Nachricht geholt werden, da die
Pipe geschlossen ist, wird Nulloption zurückgegeben. Mit einem optionalen Wert als Rückgabewert wird im Programm deutlich ausgedrückt, dass das Empfangen einer Nachricht
nicht unbedingt erfolgreich sein muss.\\

\noindent\hrulefill\par
\begin{minipage}{\linewidth}
\begin{lstlisting}[language=C++, caption=Schließen der Pipe ... void close()]
std::lock_guard pipe_lck{pipe_mtx};
open = false;
receiving_finishable.notify_all();
\end{lstlisting}
\end{minipage}

Das Schließen der Pipe mit \verb|void close()| ist deswegen interessant, weil es auch das Mutex der Pipe sperrt, obwohl man auf den ersten Blick meinen könnte, dass dies
unnötig ist: Niemand schreibt auf die Variable \verb|open| mit \verb|true| und früher oder später würden alle sehen, dass die Pipe geschlossen ist.
Der Grund, dass close() das Mutex sperren muss, liegt bei der Condition Variable. receive() verwendet die Condition Variable \verb|receiving_finishable| zur Überprüfung sowohl auf
empfangsbereite Nachrichten, als auch auf das Schließen der Pipe, denn in beiden Fällen will man nicht mehr warten. Würde close() das Mutex nicht sperren, bevor es mit der
Condition Variable alle Wartenden benachrichtigt, könnte es passieren, dass ein Prozess gerade zufällig überprüft, ob er mit dem Warten aufhören kann, weil das Mutex isst ja
nicht gesperrt, also kann er das, während close() die Benachrichtigung an alle schickt. In diesem Fall würde dieser eine Prozess die Benachrichtigung verpassen und so
im wartenden Status verbleiben. Aus diesem Grund muss close() auch das Mutex sperren.
\\

Damit haben wir uns nun einen kurzen Überblick über ein paar wichtige und interessante Teile des Programms verschafft.
