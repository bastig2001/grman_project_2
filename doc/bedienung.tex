\section{Bedienung}

Es gibt viele verschiedene Möglichkeiten die Applikation zu bedienen. Parameter kann mittels CLI, Umgebungsvariablen oder einer Konfigurationsdatei in \textsl{JSON} bestimmen.
Weiters kann man, sobald das Programm gestartet ist, über eine Kommandozeile noch zusätzlich mit dem System interagieren. Die genauen Details zu all diesen Interaktionsmöglichkeiten
lassen sich in der \textit{README} nachlesen. Hier wird nun die grundsätzliche Bedienung überblicksmäßig beschrieben.

Zur Synchronisation eines Verzeichnisses muss man über die Konsole in dieses Verzeichnis navigieren und von dort aus das Programm \verb|sync| starten. Das Programm übernimmt
beim Starten das derzeitige Arbeitsverzeichnis als Synchronisationsverzeichnis. Beim Starten kann man festlegen, ob man einen Synchronisations-Server, einen Synchronisations-Client 
oder beides haben will. Um einen Server zu starten reicht das Kommandozeilenargument \verb|-s|, dies startet einen Server der auf allen Eingangsadressen auf Port 9876 hört.
Um einen Client zu starten, muss man das Kommandozeilenargument \verb|-a| und eine Adresse oder Hostnamen angeben, der Client verbindet sich dann zu diesem Host auf Port 9876 und 
startet die Synchronisation. Um beide als ein Systemprozess zu starten, muss man beide Kommandozeilenargumente angeben. Der Server läuft solange bis man ihn nicht abbricht
und der Client macht nach dem die Synchronisation vollendet ist nichts mehr, muss allerdings auch manuell abgebrochen werden.

Beim Synchronisationsverzeichnis sollte sicher gegangen werden, dass das Programm die entsprechenden Rechte hat, um alle Dateien zu lesen und zu beschreiben.
Im Synchronisationsverzeichnis wird außerdem noch das Unterverzeichnis \textit{.sync} erstellt, in welches man lieber keine Dateien speichern sollte, da dieses Unterverzeichnis 
die Datenbankdatei für die Applikation enthält und sämtliche temporären Dateien zum Aufbau der neueren Versionen auch hier erstellt und beschrieben werden. Außerdem werden Dateien
im Unterverzeichnis \textit{.sync} nicht synchronisiert.
