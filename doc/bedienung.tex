\section{Bedienung}
\label{cha:Bedienung}

Es gibt viele verschiedene Möglichkeiten die Applikation zu bedienen. Parameter können mittels CLI, Umgebungsvariablen oder einer Konfigurationsdatei im
Format \textsl{JSON} bestimmt werden. Weiters kann man, sobald das Programm gestartet ist, über eine Kommandozeile noch zusätzlich mit dem System interagieren.

Zur Synchronisation eines Verzeichnisses muss man über die Konsole in dieses Verzeichnis navigieren und von dort aus das Programm \verb|sync| starten. 
Das Programm übernimmt beim Starten das derzeitige Arbeitsverzeichnis als Synchronisationsverzeichnis. Beim Starten kann man festlegen, ob man einen 
Synchronisations-Server, einen Synchronisations-Client oder beides haben will. Um einen Server zu starten reicht das Kommandozeilenargument \verb|-s|.
Dies startet einen Server der auf allen Eingangsadressen auf Port 9876 hört. Um einen Client zu starten, muss man das Kommandozeilenargument \verb|-a| 
und eine Adresse oder Hostnamen angeben, der Client verbindet sich dann zu diesem Host auf Port 9876 und startet die Synchronisation. 
Um beide als einen Systemprozess zu starten, gibt man beide Kommandozeilenargumente an.

Beim Synchronisationsverzeichnis sollte sicher gegangen werden, dass das Programm die entsprechenden Rechte hat, um alle Dateien lesen und beschreiben zu können.
Im Synchronisationsverzeichnis wird außerdem noch das Unterverzeichnis \textit{.sync} erstellt, in welches man lieber keine Dateien speichern sollte, da dieses 
Unterverzeichnis die Datenbankdatei für die Applikation enthält und sämtliche temporären Dateien zum Aufbau der neueren Versionen hier erstellt und beschrieben 
werden. Außerdem werden Dateien im Unterverzeichnis \textit{.sync} nicht synchronisiert.

\subsection{Konfiguration}
\label{sec:konfiguration}

Das Programm \verb|sync| kann wie bereits erwähnt auf verschiedene Weisen konfiguriert werden.
Die konkrete Hierarchie, in der die Konfigurationswerte angewendet werden, sieht wie folgt aus:
\begin{enumerate}
\item Es gibt \emph{Standardwerte} für die meisten Parameter. Diese werden überschrieben durch gesetzte
\item \emph{Umgebungsvariablen}, welche wiederum überschrieben werden durch die Werte in der
\item \emph{JSON Konfigurationsdatei}, wenn diese bereitgestellt wurde, und das letzte Wort hat die
\item \emph{Kommandozeilenschnittstelle} (englisch: command line interface; kurz: CLI)
\end{enumerate}

Alle konkreten Einstellungsmöglichkeiten inklusive Beschreibungen und auch Beispielen lassen sich in der \textit{README} des Projekts nachlesen.

\subsection{Kommandozeile}
\label{sec:command_line}

Sobald man die Applikation gestartet hat, bekommt man Zugriff auf eine Kommandozeile, mit welcher man zusätzlich mit dem Synchronisationssystem interagieren kann.
Diese Kommandozeile bietet alle Annehmlichkeiten, welche man sich von einer grundlegenden Kommandozeile erwartet.
So kann man zum Beispiel mit den Pfeiltasten nach oben und unten die bereits benutzten Befehle durchgehen. 
Um diese Kommandozeile zu verlassen und das Programm zu beenden, kann man entweder einen der äquivalenten Befehle \verb|q|, \verb|quit| oder \verb|exit|
oder das Tastaturkürzel \verb|Strg+D| benutzen.

Die folgende Tabelle zeigt alle Befehle, die in der Kommandozeile aufgerufen werden können, mit Beschreibungen:
\begin{center}
\begin{tabular}{ | c | p{8cm} | }
\hline 
\textbf{Befehl} & \textbf{Beschreibung} \\
\hline \hline
 \verb|h|, \verb|help|              & Gibt eine ein Hilfe aus \\ \hline
 \verb|ls|, \verb|list|             & Listet alle Dateien, die synchronisiert werden, auf \\  \hline
 \verb|ll|, \verb|list long|        & Listet alle Dateien, die synchronisiert werden, 
                                      inklusive Signatur, Größe und Zeitpunkt der letzten Änderung auf \\ \hline
 \verb|sync|                        & Startet, wenn möglich, einen Synchronisationdurchlauf mit dem Server 
                                      und ladet alle die Informationen über alle Dateien neu \\ \hline
 \verb|q|, \verb|quit|, \verb|exit| & Beendet und verlässt die Anwendung \\ \hline
\end{tabular}
\end{center}

\subsection{Protokollierung}
\label{sec:logging}

Die Ausgabe des Protokolls (englisch: Log) auf die Konsole kann mit dem Kommandozeilenargument \verb|-l| aktiviert werden und mit \verb|--log-level| kann
man einstellen, ab welchem Level (0 ... Verfolgung bis 5 ... kritisch) die protokollierten Nachrichten angezeigt werden sollen. Weiters kann man das 
Protokoll auch in eine Datei schreiben lassen. Der Dateiname für diese Datei kann mit dem Kommandozeilenargument \verb|-f| angegeben werden.
Existiert die angegeben Datei noch nicht, wird sie vom Programm erstellt. Außerdem wird die angegebene Protokolldatei beim Synchronisieren ignoriert.
Es ist empfehlenswert den Kommandozeileparameter \verb|--no-color| zu setzten, wenn man das Protokoll in eine Datei schreibt, da ansonsten
die Kontrollzeichen für die farbige Ausgabe in die Datei geschrieben werden, was das Protokoll schwerer lesbar macht.
Weitere Einstellungsmöglichkeiten für die Protokollierung und die Protokollausgabe lassen sich in der \textit{README} des Projekts nachlesen.
